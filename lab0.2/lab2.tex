\documentclass[12pt]{article}

\usepackage{fullpage}
\usepackage{multicol,multirow}
\usepackage{tabularx}
\usepackage{ulem}
\usepackage[utf8]{inputenc}
\usepackage[russian]{babel}
\usepackage{minted}

\usepackage{color} %% это для отображения цвета в коде
\usepackage{listings} %% собственно, это и есть пакет listings

\lstset{ %
language=C++,                 % выбор языка для подсветки (здесь это С++)
basicstyle=\small\sffamily, % размер и начертание шрифта для подсветки кода
numbers=left,               % где поставить нумерацию строк (слева\справа)
%numberstyle=\tiny,           % размер шрифта для номеров строк
stepnumber=1,                   % размер шага между двумя номерами строк
numbersep=5pt,                % как далеко отстоят номера строк от подсвечиваемого кода
backgroundcolor=\color{white}, % цвет фона подсветки - используем \usepackage{color}
showspaces=false,            % показывать или нет пробелы специальными отступами
showstringspaces=false,      % показывать или нет пробелы в строках
showtabs=false,             % показывать или нет табуляцию в строках
frame=single,              % рисовать рамку вокруг кода
tabsize=2,                 % размер табуляции по умолчанию равен 2 пробелам
captionpos=t,              % позиция заголовка вверху [t] или внизу [b] 
breaklines=true,           % автоматически переносить строки (да\нет)
breakatwhitespace=false, % переносить строки только если есть пробел
escapeinside={\%*}{*)}   % если нужно добавить комментарии в коде
}


\begin{document}
\begin{titlepage}
\begin{center}
\textbf{МИНИСТЕРСТВО ОБРАЗОВАНИЯ И НАУКИ РОССИЙСКОЙ ФЕДЕРАЦИИ
\medskip
МОСКОВСКИЙ АВИАЦИОННЫЙ ИНСТИТУТ
(НАЦИОНАЛЬНЫЙ ИССЛЕДОВАТЕЛЬСКИЙ УНИВЕРСТИТЕТ)
\vfill\vfill
{\Huge ЛАБОРАТОРНАЯ РАБОТА №2} \\
по курсу объектно-ориентированное программирование
I семестр, 2021/22 уч. год}
\end{center}
\vfill

Студент \uline{\it {Ханнанов Руслан Маратович, группа М8О-208Б-20}\hfill}

Преподаватель \uline{\it {Дорохов Евгений Павлович}\hfill}

\vfill
\end{titlepage}

\subsection*{Условие}
Создать класс комплексного числа в тригонометрической форме. Обязательно должны присутствовать операции сложения, вычитания, умножения, деления, сравнения и сопряженное число. Реализовать операции сравнения по действительной части. Реализовать литерал. Исходный код лежит в файле main.cpp.
Исходный код лежит в трёх файлах:
\pagebreak
\subsection*{Протокол работы}
3 60 \\
2 15 \\
Complex num in trigonometric form: 3*(cos60 + i * sin60) \\
Complex num in trigonometric form: 2*(cos15 + i * sin15) \\
\\
Complex num in trigonometric form: 10*(cos40 + i * sin40)
\\
Сложение: Complex num in trigonometric form: 4.39362*(cos-185.045 + i * sin-185.045) \\
\\
Вычитание: Complex num in trigonometric form: 2.58769*(cos-121.102 + i * sin-121.102) \\
\\
Умножение: Complex num in trigonometric form: 6*(cos75 + i * sin75) \\
\\
Деление: Complex num in trigonometric form: 1.5*(cos45 + i * sin45) \\
\\
Проверка на равенство: 0 \\
Сопряженное число: Complex num in trigonometric form: 3*(cos-60 + i * sin-60) \\
\\
Сравнение по модулю: 1 \\
\subsection*{Дневник отладки}
Проблем и ошибок при написании данной работы не возникло.

\subsection*{Недочёты}


\subsection*{Выводы}
В процессе выполнения данной лабораторной работы, я познакомился с пользовательскими литералими. Они показались мне достаточно удобными и практичными. Реализованный мной литерал позволяет быстро получить данные о комплексном числе.

\vfill
\pagebreak
\subsection*{Исходный код:}

{\Huge main.cpp}
\inputminted
    {C++}{main.cpp}
    
\end{document}