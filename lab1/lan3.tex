\documentclass[12pt]{article}

\usepackage{fullpage}
\usepackage{multicol,multirow}
\usepackage{tabularx}
\usepackage{ulem}
\usepackage[utf8]{inputenc}
\usepackage[russian]{babel}
\usepackage{minted}

\usepackage{color} %% это для отображения цвета в коде
\usepackage{listings} %% собственно, это и есть пакет listings

\lstset{ %
language=C++,                 % выбор языка для подсветки (здесь это С++)
basicstyle=\small\sffamily, % размер и начертание шрифта для подсветки кода
numbers=left,               % где поставить нумерацию строк (слева\справа)
%numberstyle=\tiny,           % размер шрифта для номеров строк
stepnumber=1,                   % размер шага между двумя номерами строк
numbersep=5pt,                % как далеко отстоят номера строк от подсвечиваемого кода
backgroundcolor=\color{white}, % цвет фона подсветки - используем \usepackage{color}
showspaces=false,            % показывать или нет пробелы специальными отступами
showstringspaces=false,      % показывать или нет пробелы в строках
showtabs=false,             % показывать или нет табуляцию в строках
frame=single,              % рисовать рамку вокруг кода
tabsize=2,                 % размер табуляции по умолчанию равен 2 пробелам
captionpos=t,              % позиция заголовка вверху [t] или внизу [b] 
breaklines=true,           % автоматически переносить строки (да\нет)
breakatwhitespace=false, % переносить строки только если есть пробел
escapeinside={\%*}{*)}   % если нужно добавить комментарии в коде
}


\begin{document}
\begin{titlepage}
\begin{center}
\textbf{МИНИСТЕРСТВО ОБРАЗОВАНИЯ И НАУКИ РОССИЙСКОЙ ФЕДЕРАЦИИ
\medskip
МОСКОВСКИЙ АВИАЦИОННЫЙ ИНСТИТУТ
(НАЦИОНАЛЬНЫЙ ИССЛЕДОВАТЕЛЬСКИЙ УНИВЕРСТИТЕТ)
\vfill\vfill
{\Huge ЛАБОРАТОРНАЯ РАБОТА №3} \\
по курсу объектно-ориентированное программирование
I семестр, 2021/22 уч. год}
\end{center}
\vfill

Студент \uline{\it {Ханнанов Руслан Маратович, группа М8О-208Б-20}\hfill}

Преподаватель \uline{\it {Дорохов Евгений Павлович}\hfill}

\vfill
\end{titlepage}

\subsection*{Условие}

Задание: \
Вариант 22: Пятиугольник, Шестиугольник, Восьмиугольник.\
Необходимо спроектировать и запрограммировать на языке C++ классы трех фигур, согласно варианту задания. Классы должны удовлетворять следующим правилам:
\begin{enumerate}
\item Должны быть названы также, как в вариантах задания и расположенны в раздельных файлах: отдельно заголовки (имя\_класса\_с\_маленькой\_буквы.h), отдельно описание методов (имя\_класса\_с\_маленькой\_буквы.cpp).
\item Иметь общий родительский класс Figure;
\item Содержать конструктор, принимающий координаты вершин фигуры из стандартного потока std::cin, расположенных через пробел. Пример: "0.0 0.0 1.0 0.0 1.0 1.0 0.0 1.0"
\item Содержать набор общих методов:
\begin{itemize}
    \item size\_t VertexesNumber() - метод, возвращающий количество вершин фигуры;
    \item double Area() - метод расчета площади фигуры;
    \item void Print(std::ostream& os) - метод печати типа фигуры и ее координат вершин в поток вывода os в формате: "Rectangle: (0.0, 0.0) (1.0, 0.0) (1.0, 1.0) (0.0, 1.0)" с переводом строки в конце.
\end{itemize}
\end{enumerate}

\subsection*{Описание программы}

Исходный код лежит в 10 файлах:
\begin{enumerate}
\item src/main.cpp: основная программа, взаимодействие с пользователем посредством комманд из меню

\item include/figure.h:    описание абстрактного класса фигур

\item include/point.h:     описание класса точки
\item include/pentagon.h:  описание класса пятиугольника, наследующегося от figures
\item include/hexagon.h: описание класса шестиугольника, наследующегося от figures
\item include/octagon.h:    описание класса восьмиугольника, наследующегося от figures

\item include/point.cpp:     реализация класса точки
\item include/pentagon.cpp:  реализация класса пятиугольника, наследующегося от figures
\item include/hexagon.cpp: реализация класса прямоугольника, наследующегося от figures
\item include/octagon.cpp:    реализация класса восьмиугольника, наследующегося от figures

\end{enumerate}

\subsection*{Протокол работы}
0 0 0 2 1 3 2 3 3 0 \\
Pentagon created via istream \\
Pentagon:(0, 0) (0, 2) (1, 3) (2, 3) (3, 0) \\
Area is 7 \\
Number of vertexes is 5 \\
0 0 0 3 2 6 5 6 7 3 7 0 \\
Hexagon created via istream \\
Hexagon:(0, 0) (0, 3) (2, 6) (5, 6) (7, 3) (7, 0) \\ 
Area is 36 \\
Number of vertexes is 6 \\
0 0 0 4 4 8 6 10 8 10 10 8 14 4 14 0 \\
Octagon created via istream \\
Octagon:(0, 0) (0, 4) (4, 8) (6, 10) (8, 10) (10, 8) (14, 4) (14, 0) \\
Area is 104 \\
Number of vertexes is 8 \\
Object octagon Octagon:(0, 0) (0, 4) (4, 8) (6, 10) (8, 10) (10, 8) (14, 4) (14, 0) \\
deleted \\
Object hexagon Hexagon:(0, 0) (0, 3) (2, 6) (5, 6) (7, 3) (7, 0) \\
deleted \\
Object Pentagon Pentagon:(0, 0) (0, 2) (1, 3) (2, 3) (3, 0) \\
deleted \\

\subsection*{Дневник отладки}


\subsection*{Недочёты}
Недочетов не заметил

\subsection*{Выводы}
В процессе выполнения данной лабораторной работы я познакомился с основами ООП и его принципами, такими как наследование, абстракция, инкапсуляция и полиморфизм. В работе также нужно было реализовать перегрузку оператора. В качестве результата я написал три класса фигур, наследуемых от одного начального класса и показал примеры их использования. 


\vfill

\subsection*{Исходный код:}

{\Huge figure.h}
\inputminted
    {C++}{figure.h}
    
{\Huge point.h}
\inputminted
    {C++}{point.h}
    
{\Huge point.cpp}
\inputminted
    {C++}{point.cpp}

{\Huge hexagon.h}
\inputminted
    {C++}{hexagon.h}
    
{\Huge hexagon.cpp}
\inputminted
    {C++}{hexagon.cpp}

{\Huge octagon.h}
\inputminted
    {C++}{octagon.h}
    
{\Huge octagon.cpp}
\inputminted
    {C++}{octagon.cpp}

{\Huge pentagon.h}
\inputminted
    {C++}{pentagon.h}
    
{\Huge pentagon.cpp}
\inputminted
    {C++}{pentagon.cpp}

{\Huge main.cpp}
\inputminted
    {C++}{main.cpp}
    \pagebreak
    
\end{document}